% GENERAL NON-MATH
\newcommand{\titlesubtitle}[2]{\title{\huge \textbf{#1} \\ \Large #2}}

\usepackage{ragged2e} % for justify
\newcommand{\floatnote}[1]{\vspace{.5\baselineskip}\begin{justify}\footnotesize \textbf{Note:} #1\end{justify}} 

\newcommand\independent{\protect\mathpalette{\protect\independenT}{\perp}}
\def\independenT#1#2{\mathrel{\rlap{$#1#2$}\mkern2mu{#1#2}}}

% Sweave-Options
\setkeys{Gin}{width=1\textwidth}
%\ifthenelse{\boolean{Sweave@gin}}{\setkeys{Gin}{width=0.6\textwidth}}{}

% DIRECT OUTPUT
%\DefineVerbatimEnvironment{sout}{Verbatim}{fontsize=\scriptsize, xleftmargin=0mm, fontshape=n} % manual use
%\DefineVerbatimEnvironment{Sinput}{Verbatim}{fontsize=\small, fontshape=n} % Sweave
%\DefineVerbatimEnvironment{Soutput}{Verbatim}{fontsize=\small, fontshape=n}
%\DefineVerbatimEnvironment{Scode}{Verbatim}{fontsize=\small, fontshape=n}

% THEOREMS
% now in extra file 

\let\oldappendix\appendix\renewcommand{\appendix}{\oldappendix \renewcommand{\thesection}{A.\arabic{section}}}

%\renewcommand{\qedsymbol}{\blacksquare}

% CALCULUS / ANALYSIS
\newcommand{\pd}[2]{\frac{\partial #1}{\partial #2}} 
\newcommand{\pdinline}[2]{\partial #1/\partial #2} 
\newcommand{\pdsq}[2]{\frac{\partial^2 #1}{\partial #2^2}} 
\newcommand{\seq}[1]{\left\{\m{#1}_t\right\}}
\newcommand{\norm}[1]{\left\lVert #1 \right\rVert}
%\renewcommand{\setminus}[1]{\backslash \left\{ #1 \right\}}

% MATRIX ALGEBRA
\newcommand{\m}[1]{\boldsymbol{#1}}
\newcommand{\mt}[1]{\boldsymbol{#1}^{T}}
\renewcommand{\v}[1]{\boldsymbol{#1}}
\newcommand{\vt}[1]{\boldsymbol{#1}^{T}}
\newcommand{\mi}[1]{\boldsymbol{#1}^{-1}}
\newcommand{\kronecker}{\raisebox{1pt}{\ensuremath{\:\otimes\:}}} 
\newcommand{\tr}[1]{\text{tr}\left(#1\right)}
\newcommand{\rk}[1]{\text{rk}\left(#1\right)}
\newcommand{\vech}[1]{\text{vech}\left(#1\right)}
\renewcommand{\vec}[1]{\text{vec}\left(#1\right)}

% STATISTICS FUNCTIONS
\newcommand{\var}[1]{\text{Var}\left[#1\right]}
\newcommand{\cov}[2]{\text{Cov}\left[#1,#2\right]}
\newcommand{\ex}[1]{\mathbb{E}\left[#1\right]}
\newcommand{\ext}[1]{\mathbb{E}\left[\left. #1 \right| \mathcal{F}_{t-1}\right]}
\newcommand{\eex}[3]{\int_{#3} #1 d#2} % x, measure, space
\newcommand{\prob}[1]{\mathbb{P}\left[#1\right]}
\newcommand{\pn}[1]{\mathbb{P}_n\left[#1\right]}
\newcommand{\normal}[2]{\mathcal{N}\left(#1, #2\right)}
\newcommand{\Op}[1]{\mathcal{O}_p\left(#1\right)}
\newcommand{\op}[1]{o_p\left(#1\right)}
\renewcommand{\det}[1]{\text{det}(#1)}


% ECONOMETRICS
\newcommand{\xb}{\mt{x}_i\m{\beta}}

% PROBABILITY MASS FUNCTIONS / CUMULATIVE DISTRIBUTION FUNCTIONS
\newcommand{\pmfnormal}[1]{\frac{1}{\sqrt{2\pi}} \exp\left(-\frac{1}{2} #1 \right)}
\newcommand{\cdflogistic}[1]{\left(1+\exp\left(-#1\right)\right)^{-1}}
\newcommand{\cdfgumbel}[1]{\exp\left(-\exp\left(-#1\right)\right)}

% MISC
\usepackage{amsopn}
\DeclareMathOperator*{\argmin}{arg\,min}
\DeclareMathOperator*{\argmax}{arg\,max}
\newcommand{\idxt}{\mathcal{I}_T}
\newcommand{\idxn}{\mathcal{I}_N}
\newcommand{\as}{\hspace{2.5ex}\textrm{a.s.}}
\newcommand{\indicator}{\mathbbm{1}}
\newcommand{\INDICATOR}{\text{\Large \raisebox{-1pt}{\ensuremath{\mathbbm{1}}}}}
\newcommand{\defeq}{\mathrel{\mathop:}=}
\newcommand{\sgn}{\text{sgn}}

\renewcommand{\bullet}{\raisebox{0.4ex}{\ensuremath{\centerdot}}} 

	
